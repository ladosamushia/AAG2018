\required{Project Summary}

\newpage

\required{Project Description}

\section{Introduction} % start page 1 % Galaxy Surveys are great Spectroscopic
galaxy surveys play an indispensable role in precision cosmology.  The
measurements of the Baryon Acoustic Oscillation (BAO) peak position at various
redshifts put tight constraints on the expansion rate and geometry of the
Universe. The Redshift-Space Distortion (RSD) signal, which has its origins in
the growth of structure over cosmic time, can constraints the growth rate as
well as expansion rate as a function of redshift. These measurements have been
pushed to a very high precision. BAO are now measured at a one per cent
precision,  while the best RSD measurements stand at about a five per cent
precision. The next generation of galaxy surveys, such as Dark Energy
Spectroscopic Instrument (DESI), Euclid satellite mission, and the Wide Field
Infrared Survey Telescope (WFIRST), will extend these measurements to higher
redshifts and even better precision.

% Dark Energy gravity still unclear close to LCDM and GR When these
measurements are combined with a Cosmic Microwave Background (CMB) data the
resulting constraints on Dark Energy (DE) and theories of gravity (TG), they
are overall consistent with the standard cosmological  $\Lambda$CDM model. The
constraints however at this point are not tight enough to conclusively rule
out alternative models of DE and TG that are arguably better motivated from a
fundamental physics point of view than the standard cosmological model. In
addition, there are a number of 2 to 3 $\sigma$ discrepancies (such as e.g.
the discrepancy with the local Hubble constant measurements, or the tension
with the clustering amplitude measured by weak lensing surveys), but they are
not statistically significant enough to make a conclusive case in favour of
non-standard models.
 
% There will be a cosmic variance limit for standard analysis Since we have
only one vintage point for observing the Universe the amount of information
that is  available to us is fundamentally limited. Both BAO and RSD tests, as
currently implemented, rely on two-point statistics measurements of the galaxy
field the precision of which is limited by the total  observable volume.
Expected DESI data will almost saturate this ``cosmic variance'' limit at
lower redshifts where DE dominates the evolution. Hopefully, we will be able
to either strongly rule out alternative scenarios or find a convincing
non-$\Lambda$CDM signal. If we, however, end up with a number of nagging
discrepancies that are suggestive but not statistically significant, the only
way to resolve the issue will be to find an alternative source of robust and
precise cosmological constraints in the same data.

% We need to squeeze out more information There are a number of ways to extract
more information from spectroscopic galaxy data. We can try to enhance the
standard BAO analysis by (by now standard) method of reconstructing the linear
field or trying to come up with more optimal weighting schemes. Other
approaches include looking at stacked voids, joint multi-tracer analysis, and
going to smaller scales by means of simulation assisted modeling. All these
approaches have their advantages and disadvantages and all of them must be
tried out on real data. In this proposal we would like to pursue the same goal
by going to the third-order statistics of galaxy fields.

% what we propose, what we ask for The main objective of this proposal is to
develop methods for \textbf{extracting robust BAO and RSD measurements from
the  bispectrum of galaxies on linear and semi-linear scales}. We argue that,
even though this is a difficult  endeavour that has a number of inherent
risks, it also has a number of potential advantages over alternatives (e.g.
using a two-point statistics of reconstructed fields, or extracting RSD signal
from stacked objects such as stacked voids), and is worth pursuing. The main
appeal of bispectrum is that, at least for some future surveys, it has a
potential to significantly enhance the standard BAO and RSD analysis and will
have more information than the reconstructed galaxy field. The main risk is
that the theoretical modeling of bispectrum may turn out to be to complicated
and the results may be limited by our inability to properly model non-linear
effects in the bispectrum. We hope to convince our colleagues that the
benefits  outweighs the risks and this line of research merits a modest
investment of resources (partial summer support for a PI and Co-PI, PhD
student and a Postdoc), and that some of the pre-existing work that we
performed in this direction places us in a good position  to perform this
research. % end page 1

\section{Bispectrum overview}

% 2pt and 3pt Spectroscopic galaxy surveys provide us with highly accurate
measurements of 3D positions of millions of galaxies in large cosmic volumes.
The specific distribution of galaxies that we observe is just one out of a
large number of possible distribution that could have been observed in the
same cosmological model if the stochastic initial conditions were slightly
different. The most popular way of describing this stochasticity in the large-
scale structure community has been to use the n-point correlation functions.
The two-point correlation function $\xi(\vec{r})$ is a function of a
separation vector $\vec{r}$ and describes a ``clustering strength'' at a
certain scale (average number of neighbours at distance $\vec{r}$ from a
randomly selected galaxy). Three-point function $\eta(\vec{r}_1, \vec{r}_2)$,
similarly, is a describes a probability of finding three galaxies  in a
specific triangular configuration. This information is sometimes expressed in
terms of Fourier transforms of these two functions  (power-spectrum
$P(\vec{k})$ and bispectrum $B(\vec{k}_1, \vec{k}_2)$ respectively). From now
on we will use the term ``bispectrum'' to generically refer to the three-point
function both in Fourier and configuration spaces. In principle, the hierarchy
of all infinite n-point functions contains all of the statistical information
encoded in the stochastic field. In practice, measuring and analyzing higher
order statistics gets increasingly difficult and at least for the current
state of knowledge and technological capabilities is probably confined to the
second and third orders.

% How is 3pt generated For purely Gaussian fields the two-point statistics
contains all of the information and the expectation value of bispectrum is
zero.  The seed cosmological fluctuations are believed to be very close to
Gaussian. There are two main mechanisms for generating non-zero bispectrum
even from perfectly Gaussian initial conditions. The main mechanism is
gravitational instability, which is non-linear.  Non-linear terms in the Euler
equation will inevitably couple initially independent Fourier modes and will
therefore generate bispectrum. The second mechanism is non-linear biasing
between Dark Matter (DM) and galaxies. The relationship between DM and galaxy
overdensities  is linear on very large scales but non-linear and non-local
effects become more important on smaller scales. Because of this non-linear
mapping the galaxy distribution would have a non-zero bispectrum even if it
was generated form a purely Gaussian DM field. This immediately suggests that
bispectrum should be sensitive to the exact nature of TG and higher order bias
parameters. We will later argue that the sensitivity to DE  through projection
effects is even stronger.

Inflationary theories predict that initial fluctuations have some level of
primordial non-Gaussianity imprinted in them. Planck data put very tight upper
limit on local primordial non-Gaussianity parameter -- $f_\mathrm{nl}$ -- at
least for some triangular configurations. This is not a problem for our
research program since we do not intend to use bispectrum as means to measure
$f_\mathrm{nl}$, in fact the small value of $f_\mathrm{nl}$ makes the modeling
part significantly easier. Our goal is to measure BAO/RSD from the distortions
of the bispectrum shape and as we will later show for this goal the bispectrum
generated by gravitational instability and higher order bias is sufficient.

% What does 3pt look like Formally, the measured bispectrum is a function of
two wavevectors, but azimuthal symmetry with respect to the line of sights
makes it into a function of five variables: three variables describing the
size of the triangle (e.g. wavelengths $k_1, k_2, k_3$ corresponding to three
sides) and two variables describing its orientation with respect to the line
of sight (e.g. angle that one of the sides makes with respect to the line of
sight, and the internal rotational angle of the triangle). The potential to
constrain cosmological parameters comes from the fact that the exact shape of
this  five dimensional function depends on initial conditions, the nature of
gravity that generated it, and the higher order bias.

% 3pt signal The accuracy of bispectrum measurements is related to the number
of independent galaxy triangles in the survey. The dominant contributions to
the uncertainty in measured bispectrum come from ``cosmic variance'' and
``shot-noise''. Instrumental effects and sub-optimal choices in estimator
implementations are usually sub-dominant when it comes to measurement
variance. Shot-noise stems from the fact that sparsely sampled galaxy
populations have smaller number of triangles and for the bispectrum (assuming
Poisson distribution) scales as  $1/n^2$, where $n$ is the number density of
galaxies. The cosmic variance scales as $1/V$ with the survey volume and
reflects the fact that larger volumes have more independent triangular
configurations. Large scale measurements are limited by the cosmic variance
and small scale measurements by the shot-noise. The overall variance of the
bispectrum measured in wavevector bins scales as (up to a normalization
factor, and under a number of simplifying assumptions)  \begin{equation}
\mathrm{Variance}\left[B(\vec{k}_1,\vec{k}_2,\vec{k}_3)\right] \propto
\frac{k_1k_2k_3\Delta k_1\Delta k_2\Delta
k_3}{V}\left[\frac{P(\vec{k}_1)P(\vec{k}_2) + P(\vec{k}_2)P(\vec{k}_3) +
P(\vec{k}_3)P(\vec{k}_1)}{n} + \frac{1}{n^2}\right], \end{equation} where
$\Delta k$s are the widths of wavevector bins. For comparison the power
spectrum variance scales as \begin{equation}
\mathrm{Variance}\left[P(\vec{k})\right] \propto \frac{k^2\Delta
k}{V}\left[P(\vec{k}) + \frac{1}{n}\right].  
\end{equation}  

This rough comparison demonstrates that the signal to noise of measured
bispectrum benefits much more from increasing sampling density  compared to
the power spectrum ($n^{-2}$ vs $n^{-1}$). It also benefits much more by
extending the analysis to smaller scales (higher wavenumbers, $k^6$ vs
$k^3$). So, the higher the number density is and the smaller the non-linear
effects are more competitive the bispectrum constraints become. Our forecasts
(presented in subsequent sections) suggest that, if we manage to control
systematic effects, bispectrum is capable of delivering impressively
competitive constraints for DESI Bright Galaxy Survey (BGS) and the lower
redshift range of WFIRST. Both of these samples have a high number density and
WFIRST is additionally at a high redshift (hopefully allowing to push the
analysis to higher values of $k_\mathrm{max}$. Although not as impressive, the
projected bispectrum constraints from DESI main samples and Euclid are also
very interesting.

\section{BAO/RSD analysis}

The primordial shape is imprinted very cleanly in the initial distribution.
Later gravitational evolution mixes the modes and moves some of the shape
information from power spectrum to bispectrum. If cosmological constraints
were coming only from the shape of the power spectrum and bispectrum the
contribution of bispectrum  would be modest. Reconstruction of the initial
linear field would then move all of this higher order information back to the
power spectrum. Fortunately, we have other effects in the observed clustering
that are far more sensitive to DE and TG then the initial shape. Most of the
BAO constraints on DE actually come from so called Alcock-Patczynski (AP)
distortions of the feature rather than actual determination of its position.
Similarly, most of TG constraints come from radial RSD signal.

% BAO/RSD in general BAO is a preferred scale set by processes in yearly
Universe. It manifests itself as a peak in the correlation function and an
oscillatory pattern in the power spectrum. The BAO scale has been measured
with an exquisite precision by CMB experiments and convincingly detected in
low redshift galaxy samples. Since baryonic matter only comprises a small
fraction of non-relativistic matter, the BAO feature is not as strong in
galaxy distribution as it is in CMB. The constraining power of the BAO in
galaxy distribution is strongly enhanced by the AP effect.

The AP effect results from the fact that we observe galaxies in the space of
angular coordinates and redshifts. To link these  to theoretical  predictions
we need to translate them into physical distances. To do this we need to know
the angular diameter distance $D_A$ (for angular distances)  and the Hubble
parameter $H$ (for radial distances) at the redshift of our galaxy sample. If
we pick wrong $D_A$ and $H$ our scaling will be off. Since we know the
position of the BAO peak at a very high precision from CMB data, we can use AP
distortions to find a correct value of $D_A$ that aligns BAO in galaxy
distribution with the correct physical scale. Performing this alignment in the
radial direction similarly constraints $H$. These $D_A$ and $H$ measurements
can then be translated into strong and robust DE constraints. In principle, AP
effect would work even without the BAO feature, but the presence of a well
calibrated peak (or for the power spectrum oscillatory pattern with known
frequency) makes it easier to detect small shifts.

RSD is an anisotropic signal in the galaxy distribution that results from the
fact that galaxies on average tend to fall into each other and because of this
coherent extra Doppler shift their radial distances seem on average to be
smaller than they really are. Since we belie that their is no fundamental
reason for galaxy distributions to have a different statistical pattern in
radial and angular directions, we can use measured RSD to constrain infall
velocities of galaxies. And since gravitational interactions source the infall
velocities this information can then be translated into strong constraints on
TG.

BAO are considered to be very robust and virtually systematics free
measurements. They are very large scale features that are not sensitive to the
effects of nonlinear growth and bias. RSD modeling is more involved and
significant systematic effects can be present already on semi-linear scales.
There is no straightforward way of isolating the RSD feature in the analysis,
so the RSD measurements must always incorporate AP measurements on the full
power spectrum shape. However, if the systematics can be brought under control
RSD enhance the constraining power of galaxy surveys by an order of magnitude,
and make it possible to constrain TG in addition to DE.

% BAO/RSD in 2pt Our flagship BAO/RSD measurements have so far come from two
point statistics. Most recently BAO has been measured in 6dF, CMASS, BOSS,
WiggleZ, and Vipers surveys. The BOSS measurement is so far the most accurate
at 1\% at $z\sim0.57$. RSD has been measured in CMASS, BOSS, WiggleZ, and
Vipers surveys. Large-scale BOSS measurements stand at 5\%. DESI, Euclid, and
WFIRST are projected to make multiple similar sub-percent level measurements
up to a redshift of $z\sim 2$.

% BAO/RSD in 3pt The full fledged equivalent BAO/RSD bispectrum measurements
on linear and semi-linear scales have not so far been performed mainly due to
technical difficulties with measurements, theoretical modeling, and
difficulties in estimating covariance matrices. Many early papers have looked
at the small scale three-point function to constrain second order bias []. The
large scale BOSS bispectrum monopole has been used in conjunction with the
power spectrum to enhance RSD measurements. BAO has been detected in the BOSS
large scale three-point function [] and the bispectrum []. The analysis of
large scale quadrupole of the bispectrum has not been performed so far to the
best of our knowledge.

\section{Bispectrum projections}

% Let's do forecast At this point, it would be interesting to check how much
cosmologically relevant information is in the potential bispectrum measurements
from future surveys. We use conventional Fisher information matrix techniques
to perform these forecasts. We assume that bispectrum is analysed in exactly
the same way as the full BAO/RSD power spectrum analysis, i.e. it is measured
in bins that are narrow enough not too loose a lot of information to in-bin
averaging, the nonlinear effects are separated by choosing a value of
$k_\mathrm{max}$ appropriate for each redshift (more scales are included for
higher redshifts), uncertainties in the DM to galaxy biasing and intra-halo
motions are parametrized by bias parameters and a fingers of god dispersion
term. The parameters of interest are the two AP distortion parameters that
correspond to $D_A$ and $H$ and an RSD parameter ($f\sigma_8$) that can be
related to the growth rate.

% Forecasts are good  We expect bispectrum to perform well on densely sampled
surveys. Figure 1 shows such prediction for ($f\sigma_8$) parameter for DESI
survey divided in redshift bins of $\Delta z \sim 0.1$. Plotted is the ratio of
$f\sigma_8$ variance from power spectrum to that of the bispectrum. We expect
bispectrum to perform better for denser samples (such as e.g. DESI BGS). It is
interesting and encouraging to see that the bispectrum analysis outperforms the
standard power spectrum in this regime by almost a factor of 3!

At higher redshifts the improvement is more modest but even there the
bispectrum RSD constraints are comparable to the ones derived from power
spectrum. It is worth highlighting that these predictions are for the power
spectrum only. Combining $P(\vec{k})$ and $B(\vec{k}_1,\vec{k}_2)$ Fisher
matrices is complicated because of the presence of cross-correlation terms and
we currently do not have a reliable code to perform these computations.
Nevertheless, it is clear that the addition of the bispectrum RSD would
strongly enhance overall constraints even accounting for the correlations.

\section{Bispectrum advantages}

In this section we will compare bispectrum analysis to alternative ways of
extracting extra information from galaxy surveys on top of standard two-point
BAO/RSD. The intent here is not to show that the bispectrum approach is
superior but rather to argue that despite its intrinsic difficulties it has
some interesting advantages and merits further development and research.

% Bispectrum over improvements
A safe option (one that is not prone to introducing additional systematics) for
improving BAO/RSD measurements from the power spectrum is to weighting schemes
that are in some ways more optimal than the standard Feldman-Kaiser-Peacock
prescription. Weighting schemes based on relative bias of galaxy sub-samples and
redshift evolution have recently been studied in literature. Even though these
weighting schemes have a potential to somewhat improve the measurements they
could never achieve a factor of three improvement that is potentially in the
bispectrum.

% Bispectrum over reconstruction
A very popular methods, that has by now become standard, is to undo some of the
effects of non-linear evolution on large scales by reconstructing the initial
field. This sharpens the power spectrum shape around BAO and makes it more
sensitive to the AP test (sensitivity to small distortions scales as
$\mathrm{d}P/\mathrm{d}k$. The effectiveness of reconstruction is limited by
two factors. At high redshifts the BAO shape is closer to linear and there is
not much to gain by making it sharper. Also, it is not completely clear how the
reconstruction interplays with RSD. Even though it is possible to run a
reconstruction algorithm without removing RSD signal, some kind of RSD modeling
will have to be adopted, and it is not entirely clear that this procedure will
not bias the extracted RSD constraints. Reconstruction is therefore suitable
for BAO only constraints but does not help with the RSD analysis.

Bispectrum is an independent (although somewhat correlated) measurement that as
we argued can be used for the AP and RSD measurements. Small distortions in the
bispectrum shape can be used to constraint $D_A$ and $H$ in a way that is
identical to the power spectrum (even though the modeling of former can be more
complicated). While reconstruction linearises BAO peak in power spectrum it
also reduces the bispectrum amplitude. For the number densities that are
typical for current surveys (order of $10^{-4}\ \mathrm{Mpc}$/h) the gain in
the power spectrum sensitivity to AP is approximately equal to the loss in the
bispectrum sensitivity. For current surveys this implies that the BAO
constraints form reconstructed power spectrum are roughly equal to the kind of
constraints that are obtainable from a joint power spectrum and bispectrum
analysis, and since the former is simper there is no practical need for the
later. This is not however true at higher densities and higher redshifts where
the sharpening of power spectrum BAO feature does not gain as much information
as adding a bispectrum function to the analysis. 

In summary, the information content of reconstructed power spectrum is not
always equal to the information content of joint power spectrum and bispectrum
(and all the other higher order correlators) analysis. They would be equal for
a field in a box, where the shape of the power spectrum of initial nearly
Gaussian field contains all the information and gravitational evolution only
serves to couple the phases and dilute this information into higher order
terms. Our constraints however are coming from AP distortions of observed
quantities and having an extra distorted function (bispectrum) is in some cases
more profitable than having a slightly ``sharper'' power spectrum. This is easy
to see for a hypothetical case of a universe without a BAO feature (perhaps a
Universe with trace amounts of baryonic matter). In that universe
reconstruction would not really add anything to the power spectrum, it would
simply slightly tilt the power law. But having an extra function (bispectrum)
for the AP analysis would obviously increase the constraining power.

% Bispectrum over voids
Another popular technique for going beyond standard BAO/RSD analysis is to use
stacked voids (or clusters) that are assumed to be isotropic in physical space
because of the statistical homogeneity. The observed anisotropies than are
generated by AP and RSD and can be used to obtain DE and TG constraints. The
method has a huge statistical promise, the number of objects after all scales
as $1/V$ with volume. The main challenge is the modeling. Voids (and clusters)
are small scale objects that are strongly affected by non-linear evolution and
do not lend themselves to perturbative treatments like large scale n-point
statistics do.

% Bispectrum over small scale
Yet another extremely promising option is to use statistical measures on small
extremely non-linear scales. These scales are too non-linear to be modeled from
first principles and therefore theoretical modeling will have to be aided by
high quality and resolution cosmological simulations. The main risk factor is
whether we will in fact have suitable (in terms of quality and numbers)
simulations that are accurate enough for this purpose, large enough to have
appropriately small errorbars, and diverse enough to meaningfully cover large
parameter space which, in addition to cosmological parameters, should now
include extra parameters describing small scale physics (Halo-galaxy connection
parameters, baryonic effects, environmental effects, etc.)

% All must be tried
All these are excellent ways of significantly enhance cosmological information
coming from spectroscopic galaxy surveys, and the community hopes that they
will all be mature enough in time to be applicable to DESI/Euclid/WFIRST data.
They should all be pursued to make sure that we have multiple complementary
ways of looking at the data and spot possible systematics. It is clear that
BAO/RSD from bispectrum on linear and semi-linear scales has it's role in this
joint effort. Some potential advantage are that, unlike voids, the modeling is
bound to work at least on extremely large scales where the perturbative
approach will eventually work (the real question is how far we can slides this
boundary down the scale ladder); Unlike small scale clustering the modeling
will not rely as much on simulations (although the validation most definitely
will, and some calibration on simulations may be necessary); And compared to
reconstruction it has a theoretical potential to deliver significantly stronger
enhancements.

\section{Bispectrum -- Status Quo}

% previous works
Early bispectrum measurements in galaxy distribution have been made in
(exhaustive list of early works here with short description).

% BOSS and WiggleZ work
More recent bispectrum measurements have been performed in WiggleZ and BOSS
galaxy samples. (exhaustive list of more recent works here with short
description).

% our work
Our group has published a number of recent papers related to galaxy bispectrum
analysis. In Ref.~[] we studied the ways of reducing bispectrum information
without sacrificing information content and identified three harmonic angular
modes that contain most relevant information. In Ref.~[] we detected a BAO
signature in BOSS CMASS bispectrum monopole. Our measurement was systematics
limited but still allowed us to convert the measurements to a 2 per cent
distance measure. We showed on the mocks that a systematics free measurement
would be slightly tighter compared to the one obtainable from the reconstructed
power spectrum. We are currently working on extending our work to anisotropic
bispectrum analysis. This will allow us to constrain both angular and radial
BAO parameters. Preliminary MCMC chains from this measurement are displayed on
Figure~2. Our proposed work would take over from here and would add RSD to the
mix. We will also work on a more careful study of possible theoretical and
observational systematic effects to bring the errorbars down.


\section{Research Plan}

% What we will do
Our overall goal is to develop methods of measuring BAO and RSD in large-scale
bispectrum of galaxy surveys. Achieving this goal schematically involves five
steps: measuring bispectrum, estimating covariance matrices, modeling or
removing observational systematics, theoretical modeling, and likelihood
analysis. Our tentative plans for each of these directions are briefly
described below.

\subsection*{measurements}

The bispectrum measurements are generally more difficult and computationally
demanding than the power spectrum measurements. Fortunately, recent
developments have made this task significantly easier. Recent works have
demonstrated that even wide-angle bispectrum measurements can be reduced to a
series of Fast Fourier Transforms (FFT), and that the triangle counting can
also be reduced  to a combination of FFTs and convolutions. Our group has
developed a fast GPU implementation of bispectrum multipole algorithm that has
been validated on controlled mocks and can process order of few thousand mocks
(of the size of BOSS CMASS sample) in order of few hours.

This part of the work will be concerned with further high precision testing of
our pipeline. We will produce particle distributions with known input power
spectrum and bispectrum following e.g. Ref~[]. We will then check that our
pipeline accurately reproduces the input. To accomplish this task we will not
need to evolve the initial conditions under gravity, which will make the task
computationally inexpensive. 

In addition, we will test methods of modeling the shot-noise and window effects
on the measurements. We expect the simple Poisson shot-noise model to work
sufficiently well on large scales but this expectation will be tested on
controlled simulations. In previous works the window effects have either been
ignored or modeled in an ad-hoc way. Exact window convolution for bispectrum
requires computing a six-dimensional double convolution integrals which is
computationally challenging, especially considering the fact that the
convolution will have to be performed for every model in Monte Carlo Markov
Chains. We will investigate approximate methods that are accurate enough for our
purposes and fast enough to be implement in the likelihood pipeline. 

Another topic that we will look into is related to the optimal reduction of
bispectrum data. Measuring bispectrum monopole and quadrupole in fine
wavelength bins will result in a very big data vector. This may prove
problematic at a later stage for estimating covariance matrices. There are
multiple possible ways of reducing the data vector size. One possibility is to
identify the triangular shapes that carry the most information on BAO/RSD and
remove the others. A more elaborate method along these lines would consist of
identifying the most informative linear combinations of bispectra bins. We will
explore all these options and will check how well they perform.

\subsection*{covariances}

The covariance matrices of measured bispectra provide another challenge. The
standard method of estimating covariance matrices from a sample variance of a
large number of mocks is complicated by the fact that there are a very large
number of bins and an accurate estimation of covariance and precision matrices
will require a large number of mocks. This problem becomes even more acute for
a joint power spectrum and bispectrum analysis. The covariance matrices in this
case are large and clearly non-diagonal.

We plan to tackle this problem from two complementary directions. One is to
somehow reduce the size of data vector so that computing covariances from the
mocks becomes a possibility. Another route is to estimate covariances at least
partially theoretically and calibrate theoretical estimates with mocks. 

% reducing size
Ref.~[] showed that it is possible to compress measured bispectrum to a
significantly smaller vector without loosing too much information. This
approach is very effective but requires the knowledge of covariance matrix and
is model dependent. Other approaches include further reduction of bispectrum
multipoles into e.g. first few multipoles in triangular configuration angle, or
identifying few principal components and only using them in the analysis. All
these approaches can be tested in a straightforward manner on mocks.

% theory
The exact theoretical estimation of bispectrum covariance would require
computing a six-point function and is probably an unachievable goal. It is
however likely that, especially at high redshifts, the covariance is dominated
by the leading order terms which are reasonably easy to compute. The biggest
effect would than be the effect of the mask on covariance and the coupling
between shot-noise and mask effects. We will borrow some recent ideas that were
demonstrated to work for the power spectrum (Refs.~[]) and will generalize them
for the bispectrum while making additional simplifications (e.g. assuming that
the bispectrum window function isotropic). Taking into account the fact that
fractional error in bispectrum in individual bins is lower than for the power
spectrum and that our requirement on the accuracy of covariance matrix is not
as stringent as a similar requirement on the model, we will identify
approximations that are good enough for our purposes and do not bias likelihood
fitting strongly. We can then use methods such as shrinkage to combine our
theoretical covariance with the sample variance. For this approach to work we
do not really need parameter free covariance matrices. Presence of few extra,
physically motivated, parameters would be acceptable and they would later be
calibrated on simulations.

\subsection*{observational systematics}

All observational systematics affecting the standard two-point BAO/RSD analysis
are also likely to affect the bispectrum. Most observational systematics affect
one point distribution (e.g. certain areas of sky have a lower efficiency of
having galaxies detected or the efficiency is a function of emission line
flux). These systematic inefficiencies are usually expressed either in terms of
visibility cubes (implemented in practice as random catalogs) or as sets of
per object weights. The mitigation of observational systematics is a very high
priority task for DESI/Euclid/WFIRST collaborations. We will closely follow
this work (PI Samushia is actually involved with some of this work) and will
implement the known mitigation techniques in our bispectrum analysis. It is
difficult to imagine observational systematics that would be either exclusive
to the bispectrum or significantly more severe.

% bitwise weights
There are some systematics that may not directly translate from the power
spectrum analysis and may require a separate investigation. These are the
systematics that affect the configuration space and can not be split into a
product of one-point weights. A good example of this type of systematics is
e.g. fiber assignment effects in DESI and slitless-spectroscopy effects in
Euclid.

Two proposed methods of dealing with fiber assignment effects are to remove the
most affected modes or to use a pair-wise weighting. Even though the power
spectrum specific versions of these methods can not be directly applied to
bispectrum they can be easily generalised. It should be reasonably
straightforward to identify the most affected bispectrum configurations and to
exclude them from the analysis. It should also be reasonably straightforward
to generalise pair-wise weighting schemes to triplet-wise weighting. In fact,
the bit-wise weights as implemented in Ref.~[] can be very easily translated
into triplet-wise weights.

In summary, our main concern here is to make sure that we are on top of ongoing
systematics work and to generalise and validate developed methods for higher
order statistics.

\subsection*{modeling}

Accurate modeling of various nonlinear effects that contribute to higher order
clustering is arguably the most important step towards deriving competitive
BAO/RSD constraints form real data. The modeling can be schematically divided
into three semi-independent steps: modeling non-linear gravitational evolution
of DM, nonlinearities in halo-to-DM and galaxy-to-halo relationship, and
nonlinearities in redshift-to-real space mapping.

% nonlinear
Nonlinear evolution of DM field is an extensively studied subject. Pioneering
works have shown that perturbative approaches work on large scales. The rate of
convergence of standard perturbation theory (STP) is slow, which prompted a
development of alternative approaches including Lagrangian perturbation theory,
closure theory, and various resummation and renormalization schemes.
Perturbation theory for bispectrum is generally more challenging because of
numerical issues associated with higher dimensional integrals and even slower
convergence than for a second order statistics. The nonlinear effects in DM
growth however are expected to become milder with redshift. For high redshift
samples form DESI/Euclid/WFIRST the modeling of pure DM nonlinearities will be
a significantly easier task. We plan to work on a project similar to Ref.~[]
where the outcomes of various perturbation theory approaches will be tested on
simulations on the same footing and a careful test of validity will be
performed and a safe minimal scale will be determined for all models.

% bias
The modeling of halo and galaxy bias is complicated and does not necessarily
become easier at higher redshifts. The hope is a few parameter models (e.g.
linear and second order bias to describe halo biasing supplemented with
Fingers-of-God dispersion term to describe the interhalo motions of galaxies)
will be found sufficiently accurate on large scales. We plan to work on a
project similar to Ref.~[] and test various biasing schemes on simulations.
Following Ref.~[] we will measure the DM and halo clustering in real space and
will use them to check how well various biasing schemes work and to what extent
is inclusion of higher order bias and non-local bias necessary. The main goal
of our project is to derive unbiased BAO/RSD constraints from linear scales
which makes our job slightly easier in a sense that bias models that do not
necessarily accurately describe the physics of halo-DM and galaxy-halo
interaction may be good enough for us as long as they do not systematically
bias our BAO and RSD results.

% RSD
RSD effects do not respond well to perturbative approaches. The Most popular
methods for modeling them belong to a family of ``streaming'' models that
describe redshift-to-real space mapping geometrically without relying on
linearisation. Unfortunately, one has to refer to approximate perturbative
schemes at some point since the streaming models require knowledge of velocity
field statistics (e.g. probability distribution of infall velocities). Even so,
the streaming models have been demonstrated to work much better than plain
perturbation theory and most recently have been used to model BOSS data. One
popular model along these lines is the TNS model that has been generalised to
bispectrum. Ref.~[] showed that the model works well up to scales of
$k_\mathrm{max} \sim 0.2$ on low resolution simulations. We envision two
possible projects here. One is to test this Fourier space bispectrum model at
higher precision level demanded by future surveys and to check whether it works
for lighter halos that will host most of DESI/Euclid/WFIRST galaxies. The
second direction is to generalize a real-space work of Ref.~[] to higher
orders. We will start with the simplest approximation in which the real space
and redshift space three-point functions are related by a triple infall
velocity distribution convolution which is multivariate Gaussian. We will be
able to test how well this approximation works (and on what scales) by directly
measuring three-point infall velocity distribution in simulations and
convolving it with a measured three-point function. We will then try
probability distribution functions of increasing complexity and check if any of
them work better and at the same time have a conveniently parametrizable shape.

\subsection*{likelihood}

Fitting model to data will result in a likelihood surface in $D_A$, $H$,
$f\sigma_8$, and nuisance parameters. These measurements in turn can be used to
derive DE and TG constraints in exactly the same way as for the power spectrum.
The treatment of likelihood will be standard and we do not expect it to be
significantly different from the standard BAO/RSD analysis. 

The only potential open question is combining these measurements with the power
spectrum BAO/RSD. In the ideal case one would like to perform a joint fit to
power spectrum and bispectrum. In this case the joint BAO/RSD constraints would
internally account for all the cross-correlations. This may turn out to be
complicated because of the need for larger covariance matrices. The covariance
of bispectra and power spectra individually is significantly easier to model.
An alternative approach is to get the BAO/RSD correlations between power
spectrum and bispectrum by running fits to a suit of mocks and then combining
measurements based on empirical correlations. This approach will work as long
as the two measurements are consistent. The amount of cross-correlation will
be driven by the shot-noise and volume and is not likely to be very sensitive
to cosmological parameters.

\subsection*{Simulations}

Many aspects of the research plan will require two types of high quality
simulations. The first type is where the input bispectrum and power spectrum of
the mocks are not necessarily very accurate but are known to high accuracy. We
already have a validated pipeline for creating distributions with known
anisotropic power spectrum and bispectrum based on the ideas put forward in
Refs.~[]. These types of simulations are non-expensive to produce. We already
have order of 10,000 of them for our ongoing work and could easily increase the
number if required. We will use these simulations to test our measuring
algorithms, and effects of the mask on measurements and covariances. 

Testing theoretical models will require high quality N-body simulations that
can be used to link underlying cosmological parameters to measured bispectrum.
There are number of publicly available N-body simulations that are suitable for
this task. (exhaustive list of public N-body simulations). The variety in
sampled cosmological parameters, mass resolutions, and total volume will be
ideal for testing the robustness of theoretical models.  E.g. the GLAM
simulations have a cumulative volume of more than 2,000 Gpc$^3$ and a mass
resolution of few $10^{10}$ solar masses, which should be enough to resolve
all but the lightest halos of interest to us. These boxes are 1 Gpc per side
which will induce some super-sample effects but they have been demonstrated to
work well statistically for two-point functions and we have no reason to
believe that the effect will be large for the three-point function on the
scales of interest to us.

\subsection*{Application to real data}

We will be applying our methods to existing BOSS data as they develop. Even
though because of lower number density we do not expect the results to be
competitive, BOSS is a very well studied highly complete survey and the
application of methods and testing outcomes will have a great value. 

DESI is on schedule to start collecting data in late 2019. PI Samushia is a
collaboration member and can get the PhD student and the Postdoc access to
collaboration tools and data. By the time our methods are well tested on
simulations we will be able to apply them to DESI early data. Main DESI samples
are expected to give better results than BOSS. They will also be easier to
model because they are at a higher redshift and more linear (at least the DM
fields). We expect very strong constraints from DESI BGS because of its
extremely high sampling density. The challenge there is that the low redshift
sample will be more susceptible to nonlinear effects. If we manage to calibrate
our models to required accuracy there is a potential to more than double
standard BAO/RSD results form DESI BGS.

Eventually, after the project is completed, the methodology will be applied to
Euclid/WFIRST samples. Lower redshift range of WFIRST, which also has a very
high sampling density, can potentially be very strongly enhanced by the
bispectrum analysis.

\section{Research Team}

Our core team will consist of two senior (PI and Co-PI) and two junior (PhD
student and postdoc) members. PhD student (supervised by PI) will be working
mostly on the topics related to bispectrum estimators, and observational
systematics. The Postdoc (jointly supervised by PI and Co-PI) will lead the
efforts on theoretical modeling of bispectrum measurements and covariances.
These are of course very tentative plans. All team members will be working
collaboratively on all aspects of the project and the exact distribution of
work will organically emerge from this collaborative process. 

In addition, we will involve two undergraduate students in the research.
Undergraduate students will be employed on part time basis over school period
and will help in the software pipeline development. PI already has experience
of involving undergraduates with appropriate programming skills in this type of
research. E.g. physics majors James Minton and David Coria have worked with PI
to develop a set of tools for analysing N-body simulation data (power-spectrum,
mass functions, etc.) and REU student Peter Klinge has similarly worked with
the PI on developing codes for fast likelihood sampling. The students will be
selected from Computational Physics class that PI is teaching every year. This
will at the same time benefit the core team and help the students get a
hands-on physics research experience.

%\section{Outreach Plan}

% Previous experience Bharat

% Previous experience Lado

% Bharat's staff

% Elementary School visits

% Undergraduate research

\section{Relevant Prior Research}

% General survey staff 
PI Samushia has an extensive experience of working with spectroscopic galaxy
survey data. He was involved with the BOSS survey and led a number of key
projects related to the BAO and RSD analysis of two-point statistics. He has
also worked on systematic mitigation techniques including wide-angle effects
[], mode-nulling techniques [], and selection function related systematics [].

% Bispectrum forecasts

% Bispectrum BAO

% Quadrupole 

% Bharat

\section{Known Risk Factors}

% Measurements
Point: we have codes already.
% Covariances

% Theoretical Modeling
Unavoidable systematics in modeling. Argument: will eventually work on large
scales

% Computational Resources
Computationally prohibitive run-times

% Observational Systematics 
Another possible risk factor is that observational
systematics associated with specific surveys (DESI, Euclid, WFIRST) may make
the application of theory to data difficult. 

\section{Other research activities}

PI and Co-PI's other research activities are strongly synergistic to the
proposed program.

PI Samushia is partially funded by DOE to work on large-scale structure
catalogue creation and observational systematics mitigation (one month of
summer support and half-time PhD student). PI Samushia is also funded by NASA
for his involvement in Euclid and WFIRST experiments (cumulative two weeks of
summer support). Within Euclid he is working on sample selection algorithms
and observational systematics. He is also part of one of the WFIRST Science
Investigation Groups. Involvement in these activities makes PI Samushia
familiar with  the internals of the DESI/Euclid data and up to date with
employed systematics mitigation techniques. A know-how that will aid our
research group in later parts of the project related to applying  developed
methods to real data. This project would fund two weeks of PI Samushia's
summer research that will be spent on supervising the work of a PhD student
and a Postdoc, and involvement in software pipeline development for the
project.

% Bharat's staff
PI Ratra is a world renown expert on DE modeling. His most recent works include
analysis of \ldots

\required{Previous NSF Support}

PI Samushia has not previously been funded by the NSF.

% Bharat's staff
Co-PI Ratra has been involved in  KSU QuarkNet activities funded by a subaward
from the University of Notre Dame via their NSF award \QuarkNet", NSF grant
number PHY-1219444 (09/01/2012 -- 08/31/2018).  Cumulative KSU QuarkNet funding
over 05/01/2003|12/31/2018 is \$281,910. KSU QuarkNet activities have not
resulted in publications. 

KSU QuarkNet students and teachers have used Fermilab-loaned cosmic-ray muon
detectors to: record the cosmic-ray muon flux over a large part of Kansas and
in Kansas City, MO, for many years in some cases; study the efficacy of
different shielding material for (and become finalists in) an international
NASA high-school competition to research shielding material for a manned
mission to Mars; and study the effect of the Moon blockage on muon emission
rates from the direction of the Sun by making measurements in the direction of
the Sun, before, during, and after the 2017 August 21 Solar-eclipse totality.

The PI is responsible for seven full days of teacher-centered QuarkNet
activities each year. Typically more than 15 teachers visit at a time and
typically over 100 high school students take part. Teachers are exposed to many
different physics activities, recent ones have included classes or workshops on
statistics and error analysis; electromagnetic waves, light, lasers, coherence,
and polarization; dark matter; gravitational waves.  In alternate years
teachers visit national laboratories for four full days of activities. In 2017
they went to LIGO in LA and in 2015 it was SURF in SD.  Feedback from teachers
has been very positive. Every year high-school students spend a day at the KSU
campus involved in hands-on physics activities and at this QuarkNet Masterclass
they analyze real CERN LHC CMS data, after which they interact via
videoconference with other groups of high-school students, typically in
Central or South America (because of time zone alignment) and experimental
particle physicists at Fermilab. KSU QuarkNet high-school students have become
KSU Physics majors. 


\required{Intellectual Merit}
% This is why your project is interesting and will help further
% knowledge in the field of mathematics. 

\required{Broader Impacts}

While working on this project we will make a broader impact on society in
multiple complementary ways.

We will train a graduate student and a postdoc. The PhD student will be
involved in most aspects 

% There are 4 kinds of broader impacts.
% 1. advance discovery and understanding while promoting teaching,
% training and learning
% 2. broaden the participation of underrepresented groups
% 3. disseminated broadly to enhance scientific and technological
% understanding
% 4. benefits of the proposed activity to society
