\required{Project Summary}

\newpage

\section{Introduction} % start page 1 % Galaxy Surveys are great Spectroscopic
galaxy surveys play an indispensable role in precision cosmology.  The
measurements of the Baryon Acoustic Oscillation (BAO) peak position at various
redshifts put tight constraints on the expansion rate and geometry of the
Universe. The Redshift-Space Distortion (RSD) signal, which has its origins in
the growth of structure over cosmic time, can constraints the growth rate as
well as expansion rate as a function of redshift. These measurements have been
pushed to a very high precision. BAO are now measured at a one per cent
precision,  while the best RSD measurements stand at about a five per cent
precision. The next generation of galaxy surveys, such as Dark Energy
Spectroscopic Instrument (DESI), Euclid satellite mission, and the Wide Field
Infrared Survey Telescope (WFIRST), will extend these measurements to higher
redshifts and even better precision.

% Dark Energy gravity still unclear close to LCDM and GR When these
measurements are combined with a Cosmic Microwave Background (CMB) data the
resulting constraints on Dark Energy (DE) and theories of gravity (TG), they
are overall consistent with the standard cosmological  $\Lambda$CDM model. The
constraints however at this point are not tight enough to conclusively rule
out alternative models of DE and TG that are arguably better motivated from a
fundamental physics point of view than the standard cosmological model. In
addition, there are a number of 2 to 3 $\sigma$ discrepancies (such as e.g.
the discrepancy with the local Hubble constant measurements, or the tension
with the clustering amplitude measured by weak lensing surveys), but they are
not statistically significant enough to make a conclusive case in favour of
non-standard models.
 
% There will be a cosmic variance limit for standard analysis Since we have
only one vintage point for observing the Universe the amount of information
that is  available to us is fundamentally limited. Both BAO and RSD tests, as
currently implemented, rely on two-point statistics measurements of the galaxy
field the precision of which is limited by the total  observable volume.
Expected DESI data will almost saturate this ``cosmic variance'' limit at
lower redshifts where DE dominates the evolution. Hopefully, we will be able
to either strongly rule out alternative scenarios or find a convincing
non-$\Lambda$CDM signal. If we, however, end up with a number of nagging
discrepancies that are suggestive but not statistically significant, the only
way to resolve the issue will be to find an alternative source of robust and
precise cosmological constraints in the same data.

% We need to squeeze out more information There are a number of ways to extract
more information from spectroscopic galaxy data. We can try to enhance the
standard BAO analysis by (by now standard) method of reconstructing the linear
field or trying to come up with more optimal weighting schemes. Other
approaches include looking at stacked voids, joint multi-tracer analysis, and
going to smaller scales by means of simulation assisted modeling. All these
approaches have their advantages and disadvantages and all of them must be
tried out on real data. In this proposal we would like to pursue the same goal
by going to the third-order statistics of galaxy fields.

% what we propose, what we ask for The main objective of this proposal is to
develop methods for \textbf{extracting robust BAO and RSD measurements from
the  bispectrum of galaxies on linear and semi-linear scales}. We argue that,
even though this is a difficult  endeavour that has a number of inherent
risks, it also has a number of potential advantages over alternatives (e.g.
using a two-point statistics of reconstructed fields, or extracting RSD signal
from stacked objects such as stacked voids), and is worth pursuing. The main
appeal of bispectrum is that, at least for some future surveys, it has a
potential to significantly enhance the standard BAO and RSD analysis and will
have more information than the reconstructed galaxy field. The main risk is
that the theoretical modeling of bispectrum may turn out to be to complicated
and the results may be limited by our inability to properly model non-linear
effects in the bispectrum. We hope to convince our colleagues that the
benefits  outweighs the risks and this line of research merits a modest
investment of resources (partial summer support for a PI and Co-PI, PhD
student and a Postdoc), and that some of the pre-existing work that we
performed in this direction places us in a good position  to perform this
research. % end page 1

\section{Bispectrum overview}

% 2pt and 3pt Spectroscopic galaxy surveys provide us with highly accurate
measurements of 3D positions of millions of galaxies in large cosmic volumes.
The specific distribution of galaxies that we observe is just one out of a
large number of possible distribution that could have been observed in the
same cosmological model if the stochastic initial conditions were slightly
different. The most popular way of describing this stochasticity in the large-
scale structure community has been to use the n-point correlation functions.
The two-point correlation function $\xi(\vec{r})$ is a function of a
separation vector $\vec{r}$ and describes a ``clustering strength'' at a
certain scale (average number of neighbours at distance $\vec{r}$ from a
randomly selected galaxy). Three-point function $\eta(\vec{r}_1, \vec{r}_2)$,
similarly, is a describes a probability of finding three galaxies  in a
specific triangular configuration. This information is sometimes expressed in
terms of Fourier transforms of these two functions  (power-spectrum
$P(\vec{k})$ and bispectrum $B(\vec{k}_1, \vec{k}_2)$ respectively). From now
on we will use the term ``bispectrum'' to generically refer to the three-point
function both in Fourier and configuration spaces. In principle, the hierarchy
of all infinite n-point functions contains all of the statistical information
encoded in the stochastic field. In practice, measuring and analyzing higher
order statistics gets increasingly difficult and at least for the current
state of knowledge and technological capabilities is probably confined to the
second and third orders.

% How is 3pt generated For purely Gaussian fields the two-point statistics
contains all of the information and the expectation value of bispectrum is
zero.  The seed cosmological fluctuations are believed to be very close to
Gaussian. There are two main mechanisms for generating non-zero bispectrum
even from perfectly Gaussian initial conditions. The main mechanism is
gravitational instability, which is non-linear.  Non-linear terms in the Euler
equation will inevitably couple initially independent Fourier modes and will
therefore generate bispectrum. The second mechanism is non-linear biasing
between Dark Matter (DM) and galaxies. The relationship between DM and galaxy
overdensities  is linear on very large scales but non-linear and non-local
effects become more important on smaller scales. Because of this non-linear
mapping the galaxy distribution would have a non-zero bispectrum even if it
was generated form a purely Gaussian DM field. This immediately suggests that
bispectrum should be sensitive to the exact nature of TG and higher order bias
parameters. We will later argue that the sensitivity to DE  through projection
effects is even stronger.

Inflationary theories predict that initial fluctuations have some level of
primordial non-Gaussianity imprinted in them. Planck data put very tight upper
limit on local primordial non-Gaussianity parameter -- $f_\mathrm{nl}$ -- at
least for some triangular configurations. This is not a problem for our
research program since we do not intend to use bispectrum as means to measure
$f_\mathrm{nl}$, in fact the small value of $f_\mathrm{nl}$ makes the modeling
part significantly easier. Our goal is to measure BAO/RSD from the distortions
of the bispectrum shape and as we will later show for this goal the bispectrum
generated by gravitational instability and higher order bias is sufficient.

% What does 3pt look like Formally, the measured bispectrum is a function of
two wavevectors, but azimuthal symmetry with respect to the line of sights
makes it into a function of five variables: three variables describing the
size of the triangle (e.g. wavelengths $k_1, k_2, k_3$ corresponding to three
sides) and two variables describing its orientation with respect to the line
of sight (e.g. angle that one of the sides makes with respect to the line of
sight, and the internal rotational angle of the triangle). The potential to
constrain cosmological parameters comes from the fact that the exact shape of
this  five dimensional function depends on initial conditions, the nature of
gravity that generated it, and the higher order bias.

% 3pt signal The accuracy of bispectrum measurements is related to the number
of independent galaxy triangles in the survey. The dominant contributions to
the uncertainty in measured bispectrum come from ``cosmic variance'' and
``shot-noise''. Instrumental effects and sub-optimal choices in estimator
implementations are usually sub-dominant when it comes to measurement
variance. Shot-noise stems from the fact that sparsely sampled galaxy
populations have smaller number of triangles and for the bispectrum (assuming
Poisson distribution) scales as  $1/n^2$, where $n$ is the number density of
galaxies. The cosmic variance scales as $1/V$ with the survey volume and
reflects the fact that larger volumes have more independent triangular
configurations. Large scale measurements are limited by the cosmic variance
and small scale measurements by the shot-noise. The overall variance of the
bispectrum measured in wavevector bins scales as (up to a normalization
factor, and under a number of simplifying assumptions)  \begin{equation}
\mathrm{Variance}\left[B(\vec{k}_1,\vec{k}_2,\vec{k}_3)\right] \propto
\frac{k_1k_2k_3\Delta k_1\Delta k_2\Delta
k_3}{V}\left[\frac{P(\vec{k}_1)P(\vec{k}_2) + P(\vec{k}_2)P(\vec{k}_3) +
P(\vec{k}_3)P(\vec{k}_1)}{n} + \frac{1}{n^2}\right], \end{equation} where
$\Delta k$s are the widths of wavevector bins. For comparison the power
spectrum variance scales as \begin{equation}
\mathrm{Variance}\left[P(\vec{k})\right] \propto \frac{k^2\Delta
k}{V}\left[P(\vec{k}) + \frac{1}{n}\right].  
\end{equation}  

This rough comparison demonstrates that the signal to noise of measured
bispectrum benefits much more from increasing sampling density  compared to
the power spectrum ($n^{-2}$ vs $n^{-1}$). It also benefits much more by
extending the analysis to smaller scales (higher wavenumbers, $k^6$ vs
$k^3$). So, the higher the number density is and the smaller the non-linear
effects are more competitive the bispectrum constraints become. Our forecasts
(presented in subsequent sections) suggest that, if we manage to control
systematic effects, bispectrum is capable of delivering impressively
competitive constraints for DESI Bright Galaxy Survey (BGS) and the lower
redshift range of WFIRST. Both of these samples have a high number density and
WFIRST is additionally at a high redshift (hopefully allowing to push the
analysis to higher values of $k_\mathrm{max}$. Although not as impressive, the
projected bispectrum constraints from DESI main samples and Euclid are also
very interesting.

\section{BAO/RSD analysis}

The primordial shape is imprinted very cleanly in the initial distribution.
Later gravitational evolution mixes the modes and moves some of the shape
information from power spectrum to bispectrum. If cosmological constraints
were coming only from the shape of the power spectrum and bispectrum the
contribution of bispectrum  would be modest. Reconstruction of the initial
linear field would then move all of this higher order information back to the
power spectrum. Fortunately, we have other effects in the observed clustering
that are far more sensitive to DE and TG then the initial shape. Most of the
BAO constraints on DE actually come from so called Alcock-Patczynski (AP)
distortions of the feature rather than actual determination of its position.
Similarly, most of TG constraints come from radial RSD signal.

% BAO/RSD in general BAO is a preferred scale set by processes in yearly
Universe. It manifests itself as a peak in the correlation function and an
oscillatory pattern in the power spectrum. The BAO scale has been measured
with an exquisite precision by CMB experiments and convincingly detected in
low redshift galaxy samples. Since baryonic matter only comprises a small
fraction of non-relativistic matter, the BAO feature is not as strong in
galaxy distribution as it is in CMB. The constraining power of the BAO in
galaxy distribution is strongly enhanced by the AP effect.

The AP effect results from the fact that we observe galaxies in the space of
angular coordinates and redshifts. To link these  to theoretical  predictions
we need to translate them into physical distances. To do this we need to know
the angular diameter distance $D_A$ (for angular distances)  and the Hubble
parameter $H$ (for radial distances) at the redshift of our galaxy sample. If
we pick wrong $D_A$ and $H$ our scaling will be off. Since we know the
position of the BAO peak at a very high precision from CMB data, we can use AP
distortions to find a correct value of $D_A$ that aligns BAO in galaxy
distribution with the correct physical scale. Performing this alignment in the
radial direction similarly constraints $H$. These $D_A$ and $H$ measurements
can then be translated into strong and robust DE constraints. In principle, AP
effect would work even without the BAO feature, but the presence of a well
calibrated peak (or for the power spectrum oscillatory pattern with known
frequency) makes it easier to detect small shifts.

RSD is an anisotropic signal in the galaxy distribution that results from the
fact that galaxies on average tend to fall into each other and because of this
coherent extra Doppler shift their radial distances seem on average to be
smaller than they really are. Since we belie that their is no fundamental
reason for galaxy distributions to have a different statistical pattern in
radial and angular directions, we can use measured RSD to constrain infall
velocities of galaxies. And since gravitational interactions source the infall
velocities this information can then be translated into strong constraints on
TG.

BAO are considered to be very robust and virtually systematics free
measurements. They are very large scale features that are not sensitive to the
effects of nonlinear growth and bias. RSD modeling is more involved and
significant systematic effects can be present already on semi-linear scales.
There is no straightforward way of isolating the RSD feature in the analysis,
so the RSD measurements must always incorporate AP measurements on the full
power spectrum shape. However, if the systematics can be brought under control
RSD enhance the constraining power of galaxy surveys by an order of magnitude,
and make it possible to constrain TG in addition to DE.

% BAO/RSD in 2pt Our flagship BAO/RSD measurements have so far come from two
point statistics. Most recently BAO has been measured in 6dF, CMASS, BOSS,
WiggleZ, and Vipers surveys. The BOSS measurement is so far the most accurate
at 1\% at $z\sim0.57$. RSD has been measured in CMASS, BOSS, WiggleZ, and
Vipers surveys. Large-scale BOSS measurements stand at 5\%. DESI, Euclid, and
WFIRST are projected to make multiple similar sub-percent level measurements
up to a redshift of $z\sim 2$.

% BAO/RSD in 3pt The full fledged equivalent BAO/RSD bispectrum measurements
on linear and semi-linear scales have not so far been performed mainly due to
technical difficulties with measurements, theoretical modeling, and
difficulties in estimating covariance matrices. Many early papers have looked
at the small scale three-point function to constrain second order bias []. The
large scale BOSS bispectrum monopole has been used in conjunction with the
power spectrum to enhance RSD measurements. BAO has been detected in the BOSS
large scale three-point function [] and the bispectrum []. The analysis of
large scale quadrupole of the bispectrum has not been performed so far to the
best of our knowledge.

\section{Bispectrum projections}

% Let's do forecast At this point, it would be interesting to check how much
cosmologically relevant information is in the potential bispectrum measurements
from future surveys. We use conventional Fisher information matrix techniques
to perform these forecasts. We assume that bispectrum is analysed in exactly
the same way as the full BAO/RSD power spectrum analysis, i.e. it is measured
in bins that are narrow enough not too loose a lot of information to in-bin
averaging, the nonlinear effects are separated by choosing a value of
$k_\mathrm{max}$ appropriate for each redshift (more scales are included for
higher redshifts), uncertainties in the DM to galaxy biasing and intra-halo
motions are parametrized by bias parameters and a fingers of god dispersion
term. The parameters of interest are the two AP distortion parameters that
correspond to $D_A$ and $H$ and an RSD parameter ($f\sigma_8$) that can be
related to the growth rate.

% Forecasts are good  We expect bispectrum to perform well on densely sampled
surveys. Figure 1 shows such prediction for ($f\sigma_8$) parameter for DESI
survey divided in redshift bins of $\Delta z \sim 0.1$. Plotted is the ratio of
$f\sigma_8$ variance from power spectrum to that of the bispectrum. We expect
bispectrum to perform better for denser samples (such as e.g. DESI BGS). It is
interesting and encouraging to see that the bispectrum analysis outperforms the
standard power spectrum in this regime by almost a factor of 3!

At higher redshifts the improvement is more modest but even there the
bispectrum RSD constraints are comparable to the ones derived from power
spectrum. It is worth highlighting that these predictions are for the power
spectrum only. Combining $P(\vec{k})$ and $B(\vec{k}_1,\vec{k}_2)$ Fisher
matrices is complicated because of the presence of cross-correlation terms and
we currently do not have a reliable code to perform these computations.
Nevertheless, it is clear that the addition of the bispectrum RSD would
strongly enhance overall constraints even accounting for the correlations.

\section{Bispectrum advantages}

In this section we will compare bispectrum analysis to alternative ways of
extracting extra information from galaxy surveys on top of standard two-point
BAO/RSD. The intent here is not to show that the bispectrum approach is
superior but rather to argue that despite its intrinsic difficulties it has
some interesting advantages and merits further development and research.

% Bispectrum over improvements
A safe option (one that is not prone to introducing additional systematics) for
improving BAO/RSD measurements from the power spectrum is to weighting schemes
that are in some ways more optimal than the standard Feldman-Kaiser-Peacock
prescription. Weighting schemes based on relative bias of galaxy sub-samples and
redshift evolution have recently been studied in literature. Even though these
weighting schemes have a potential to somewhat improve the measurements they
could never achieve a factor of three improvement that is potentially in the
bispectrum.

% Bispectrum over reconstruction
A very popular methods, that has by now become standard, is to undo some of the
effects of non-linear evolution on large scales by reconstructing the initial
field. This sharpens the power spectrum shape around BAO and makes it more
sensitive to the AP test (sensitivity to small distortions scales as
$\mathrm{d}P/\mathrm{d}k$. The effectiveness of reconstruction is limited by
two factors. At high redshifts the BAO shape is closer to linear and there is
not much to gain by making it sharper. Also, it is not completely clear how the
reconstruction interplays with RSD. Even though it is possible to run a
reconstruction algorithm without removing RSD signal, some kind of RSD modeling
will have to be adopted, and it is not entirely clear that this procedure will
not bias the extracted RSD constraints. Reconstruction is therefore suitable
for BAO only constraints but does not help with the RSD analysis.

Bispectrum is an independent (although somewhat correlated) measurement that as
we argued can be used for the AP and RSD measurements. Small distortions in the
bispectrum shape can be used to constraint $D_A$ and $H$ in a way that is
identical to the power spectrum (even though the modeling of former can be more
complicated). While reconstruction linearises BAO peak in power spectrum it
also reduces the bispectrum amplitude. For the number densities that are
typical for current surveys (order of $10^{-4}\ \mathrm{Mpc}$/h) the gain in
the power spectrum sensitivity to AP is approximately equal to the loss in the
bispectrum sensitivity. For current surveys this implies that the BAO
constraints form reconstructed power spectrum are roughly equal to the kind of
constraints that are obtainable from a joint power spectrum and bispectrum
analysis, and since the former is simper there is no practical need for the
later. This is not however true at higher densities and higher redshifts where
the sharpening of power spectrum BAO feature does not gain as much information
as adding a bispectrum function to the analysis. 

In summary, the information content of reconstructed power spectrum is not
always equal to the information content of joint power spectrum and bispectrum
(and all the other higher order correlators) analysis. They would be equal for
a field in a box, where the shape of the power spectrum of initial nearly
Gaussian field contains all the information and gravitational evolution only
serves to couple the phases and dilute this information into higher order
terms. Our constraints however are coming from AP distortions of observed
quantities and having an extra distorted function (bispectrum) is in some cases
more profitable than having a slightly ``sharper'' power spectrum. This is easy
to see for a hypothetical case of a universe without a BAO feature (perhaps a
Universe with trace amounts of baryonic matter). In that universe
reconstruction would not really add anything to the power spectrum, it would
simply slightly tilt the power law. But having an extra function (bispectrum)
for the AP analysis would obviously increase the constraining power.

% Bispectrum over voids
Another popular technique for going beyond standard BAO/RSD analysis is to use
stacked voids (or clusters) that are assumed to be isotropic in physical space
because of the statistical homogeneity. The observed anisotropies than are
generated by AP and RSD and can be used to obtain DE and TG constraints. The
method has a huge statistical promise, the number of objects after all scales
as $1/V$ with volume. The main challenge is the modeling. Voids (and clusters)
are small scale objects that are strongly affected by non-linear evolution and
do not lend themselves to perturbative treatments like large scale n-point
statistics do.

% Bispectrum over small scale
Yet another extremely promising option is to use statistical measures on small
extremely non-linear scales. These scales are too non-linear to be modeled from
first principles and therefore theoretical modeling will have to be aided by
high quality and resolution cosmological simulations. The main risk factor is
whether we will in fact have suitable (in terms of quality and numbers)
simulations that are accurate enough for this purpose, large enough to have
appropriately small errorbars, and diverse enough to meaningfully cover large
parameter space which, in addition to cosmological parameters, should now
include extra parameters describing small scale physics (Halo-galaxy connection
parameters, baryonic effects, environmental effects, etc.)

% All must be tried
All these are excellent ways of significantly enhance cosmological information
coming from spectroscopic galaxy surveys, and the community hopes that they
will all be mature enough in time to be applicable to DESI/Euclid/WFIRST data.
They should all be pursued to make sure that we have multiple complementary
ways of looking at the data and spot possible systematics. It is clear that
BAO/RSD from bispectrum on linear and semi-linear scales has it's role in this
joint effort. Some potential advantage are that, unlike voids, the modeling is
bound to work at least on extremely large scales where the perturbative
approach will eventually work (the real question is how far we can slides this
boundary down the scale ladder); Unlike small scale clustering the modeling
will not rely as much on simulations (although the validation most definitely
will, and some calibration on simulations may be necessary); And compared to
reconstruction it has a theoretical potential to deliver significantly stronger
enhancements.

\section{Bispectrum -- Status Quo}

\section{Research Plan}

% What we will do
Our overall goal is to develop methods of measuring BAO and RSD in large-scale
bispectrum of galaxy surveys. Achieving this goal schematically involves five
steps: measuring bispectrum, estimating covariance matrices, modeling or
removing observational systematics, theoretical modeling, and likelihood
analysis. Our tentative plans for each of these directions are briefly
described below.

\subsection*{measurements}

The bispectrum measurements are generally more difficult and computationally
demanding than the power spectrum measurements. Fortunately, recent
developments have made this task significantly easier. Recent works have
demonstrated that even wide-angle bispectrum measurements can be reduced to a
series of Fast Fourier Transforms (FFT), and that the triangle counting can
also be reduced  to a combination of FFTs and convolutions. Our group has
developed a fast GPU implementation of bispectrum multipole algorithm that has
been validated on controlled mocks and can process order of few thousand mocks
(of the size of BOSS CMASS sample) in order of few hours.

This part of the work will be concerned with further high precision testing of
our pipeline. We will produce particle distributions with known input power
spectrum and bispectrum following e.g. Ref~[]. We will then check that our
pipeline accurately reproduces the input. To accomplish this task we will not
need to evolve the initial conditions under gravity, which will make the task
computationally inexpensive. 

In addition, we will test methods of modeling the shot-noise and window effects
on the measurements. We expect the simple Poisson shot-noise model to work
sufficiently well on large scales but this expectation will be tested on
controlled simulations. In previous works the window effects have either been
ignored or modeled in an ad-hoc way. Exact window convolution for bispectrum
requires computing a six-dimensional double convolution integrals which is
computationally challenging, especially considering the fact that the
convolution will have to be performed for every model in Monte Carlo Markov
Chains. We will investigate approximate methods that are accurate enough for our
purposes and fast enough to be implement in the likelihood pipeline. 

Another topic that we will look into is related to the optimal reduction of
bispectrum data. Measuring bispectrum monopole and quadrupole in fine
wavelength bins will result in a very big data vector. This may prove
problematic at a later stage for estimating covariance matrices. There are
multiple possible ways of reducing the data vector size. One possibility is to
identify the triangular shapes that carry the most information on BAO/RSD and
remove the others. A more elaborate method along these lines would consist of
identifying the most informative linear combinations of bispectra bins. We will
explore all these options and will check how well they perform.

\subsection*{covariances}

The covariance matrices of measured bispectra provide another challenge. The
standard method of estimating covariance matrices from a sample variance of a
large number of mocks is complicated by the fact that there are a very large
number of bins and an accurate estimation of covariance and precision matrices
will require a large number of mocks.

We may still be able to get away with computing sample covariance if we are able
to find a suitable data reduction scheme that reduces the size of our data
vector without significantly sacrificing the information content. 

\subsection*{observational systematics}

All observational systematics affecting the standard two-point BAO/RSD analysis
are also likely to affect the bispectrum. It is difficult to imagine
observational systematics that would be either exclusive to the bispectrum or
significantly more severe.  

% bitwise weights

\subsection*{modeling}
% nonlinear

% bias

% RSD

\subsection*{likelihood}

% applications to real data

\section{Outreach Plan}

% Previous experience Bharat

% Previous experience Lado

% Bharat's staff

% Elementary School visits

% Undergraduate research

\section{Relevant Prior Research}

% General survey staff 
PI Samushia has an extensive experience of working with spectroscopic galaxy
survey data. He was involved with the BOSS survey and led a number of key
projects related to the BAO and RSD analysis of two-point statistics. He has
also worked on systematic mitigation techniques including wide-angle effects
[], mode-nulling techniques [], and selection function related systematics [].

% Bispectrum forecasts

% Bispectrum BAO

% Quadrupole 

% Bharat

\section{Known Risk Factors}


% Measurements

% Covariances

% Theoretical Modeling

% Computational Resources

% Observational Systematics Another possible risk factor is that observational
systematics associated with specific surveys (DESI, Euclid, WFIRST) may make
the application of theory to data difficult. 

\required{Previous NSF Support}

% Bharat's staff
PI Ratra \ldots

PI Samushia has not previously been funded by the NSF.

\section{Other research activities}

PI and Co-PI's other research activities are strongly synergistic to the
proposed program.

PI Samushia is partially funded by DOE to work on large-scale structure
catalogue creation and observational systematics mitigation (one month of
summer support and half-time PhD student). PI Samushia is also funded by NASA
for his involvement in Euclid and WFIRST experiments (cumulative two weeks of
summer support). Within Euclid he is working on sample selection algorithms
and observational systematics. He is also part of one of the WFIRST Science
Investigation Groups. Involvement in these activities makes PI Samushia
familiar with  the internals of the DESI/Euclid data and up to date with
employed systematics mitigation techniques. A know-how that will aid our
research group in later parts of the project related to applying  developed
methods to real data. This project would fund two weeks of PI Samushia's
summer research that will be spent on supervising the work of a PhD student
and a Postdoc, and involvement in software pipeline development for the
project.

% Bharat's staff
PI Ratra is a world renown expert on DE modeling. His most recent works include
analysis of \ldots

\required{Intellectual Merit}
% This is why your project is interesting and will help further
% knowledge in the field of mathematics. 

\required{Broader Impacts}
% There are 4 kinds of broader impacts.
% 1. advance discovery and understanding while promoting teaching,
% training and learning
% 2. broaden the participation of underrepresented groups
% 3. disseminated broadly to enhance scientific and technological
% understanding
% 4. benefits of the proposed activity to society
